\documentclass[12pt]{article}
\usepackage{amsmath, amssymb, amsthm}
\usepackage{mathtools}
\usepackage{graphicx}
\usepackage{float}
\usepackage{hyperref}
\usepackage{xcolor}
\usepackage{listings}
\usepackage{geometry}
\usepackage{algorithm}
\usepackage{algpseudocode}
\usepackage{tikz}
\usepackage{longtable}
\usepackage[section]{placeins}

% Page Setup
\geometry{a4paper, margin=1in}

% Custom Commands
\newcommand{\vecb}[1]{\mathbf{#1}}
\newcommand{\brak}[1]{\ensuremath{\left(#1\right)}}
\newcommand{\cbrak}[1]{\ensuremath{\left\{#1\right\}}}
\newcommand{\abs}[1]{\left\vert#1\right\vert}
\newcommand{\norm}[1]{\left\lVert#1\right\rVert}

\begin{document}

\bibliographystyle{IEEEtran}

\vspace{3cm}

\section*{Abstract}
This report examines the response of an RC circuit to a square wave input of time period $T$, with particular focus on three cases: $T \gg RC$, $T = RC$, and $T \ll RC$. The behavior of the output voltage across the capacitor is analyzed in each case, highlighting the effect of the relationship between the time constant $RC$ and the period of the input signal.

\section{Introduction}
RC circuits, consisting of a resistor and a capacitor in series, exhibit dynamic behavior when subjected to time-varying inputs. The response of such circuits is governed by their time constant $\tau = RC$. When driven by a square wave input, the circuit's output depends on the interplay between the time constant and the time period $T$ of the square wave. This document explores the circuit's response for three distinct scenarios.

\section{Theory}
The governing equation for the voltage $V_C(t)$ across the capacitor in an RC circuit with input voltage $V_{\text{in}}(t)$ is:
\begin{align}
    iR + \frac{q}{C} &= V_{\text{in}}, \\
    V_C &= \frac{q}{C}, \\
    RC \frac{dV_C}{dt} + V_C &= V_{\text{in}}, \\
    \frac{dV_C}{dt} + \frac{V_C}{RC} &= \frac{V_{\text{in}}}{RC}.
\end{align}
For a square wave input, $V_{\text{in}}(t)$ alternates between two levels (e.g., $V_0$ and $-V_0$) with a period $T$. The response depends on the relationship between $T$ and $RC$.

\section{Numerical Solution Using the Trapezoidal Method}
To solve the governing differential equation numerically, we employ the trapezoidal method. The Python implementation simulates the circuit response to a square wave input.\newline
According to trapezoidal rule
\begin{align}
  y_{n+1}&=y_n+\frac{1}{2}h\brak{f(t_n,y_{n}) + f(t_{n+1},y_{n+1})}\\
  f(t_n, y_n) &= \frac{1}{RC}(V_{in}(t_n) - V_{C}(y_n))\\
  V_{C}(y_{n+1})&=V_{C}(y_n)+\frac{1}{2}h (\frac{1}{RC}(V_{in}(t_n) - V_{C}(y_n))+\frac{1}{RC}(V_{in}(t_{n+1}) - V_{C}(y_{n+1})))\\
  V_{C}(y_{n+1})&=V_{C}(y_n)+\frac{h}{2RC}((V_{in}(t_n) - V_{C}(y_n))+(V_{in}(t_{n+1}) - V_{C}(y_{n+1})))\\
  (1+\frac{h}{2RC})(V_{C}(y_{n+1})) &= V_C(y_n)+\frac{h}{2RC}(V_{in}(t_n)+V_{in}(t_{n+1})- V_{C}(y_n))
\end{align}
We will get the expression
\begin{align}
	(V_{C}(y_{n+1})) &= \frac{V_C(y_n)+\frac{h}{2RC}(V_{in}(t_n)+V_{in}(t_{n+1})- V_{C}(y_n))}{1+\frac{h}{2RC}}
\end{align}
\subsection{Explanation of the Code}
The Python code above simulates the response of an RC circuit to a square wave input. Key steps include:
\begin{itemize}
    \item Defining circuit parameters such as resistance, capacitance, and time step size.
    \item Generating a square wave function for the input voltage.
    \item Implementing the trapezoidal method to numerically integrate the differential equation governing the RC circuit.
    \item Updating the capacitor voltage iteratively using past and present input values.
    \item Plotting the results to visualize the circuit response.
\end{itemize}
\section{Analysis}
\subsection{Case 1: $T \gg RC$}
When the period of the square wave is much greater than the time constant ($T \gg RC$), the capacitor has sufficient time to fully charge and discharge during each half-cycle. The output voltage $V_C(t)$ closely follows the input signal, with exponential transitions between $V_0$ and $-V_0$. The response can be expressed as:
\begin{align}
    V_C(t) = V_0 \left(1 - e^{-t/(RC)}\right) \quad \text{(during charging)},
\end{align}
\begin{align}
    V_C(t) = -V_0 \left(1 - e^{-t/(RC)}\right) \quad \text{(during discharging)}.
\end{align}

\subsection{Case 2: $T = RC$}
When the period of the square wave is comparable to the time constant ($T \approx RC$), the capacitor does not fully charge or discharge during each half-cycle. The output voltage exhibits noticeable attenuation and rounded transitions, reflecting an intermediate behavior. Inititally we will have the transient response but it will become 0 eventually

\subsection{Case 3: $T \ll RC$}
For a square wave with a period much smaller than the time constant ($T \ll RC$), the capacitor cannot respond significantly within a single cycle. The output voltage $V_C(t)$ remains nearly constant, approximating the average value of the input signal. This behavior effectively acts as a low-pass filter, smoothing the square wave into a nearly constant DC level as time progress. Initially, it will grow to a certain value and then oscillates near that level 

\section{Conclusion}
The analysis and numerical simulation confirm that the RC circuit behaves as a filter when subjected to a square wave input. Depending on the relationship between $T$ and $RC$, the circuit either follows the input closely, exhibits attenuation, or behaves as a low-pass filter. The numerical solution using the trapezoidal method effectively captures these behaviors and provides insights into the circuit's transient response.

\end{document}

