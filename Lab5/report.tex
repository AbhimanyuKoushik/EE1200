\documentclass[12pt]{article}
\usepackage{amsmath, amssymb, amsthm}
\usepackage{mathtools}
\usepackage{graphicx}
\usepackage{float}
\usepackage{hyperref}
\usepackage{xcolor}
\usepackage{listings}
\usepackage{geometry}
\usepackage{algorithm}
\usepackage{algpseudocode}
\usepackage{tikz}
\usepackage{longtable}
\usepackage{circuitikz}
\usepackage{comment}
\usepackage{MnSymbol}
\usepackage{physics}
\usepackage[section]{placeins}

\providecommand{\brak}[1]{\ensuremath{\left(#1\right)}}
\geometry{a4paper, margin=1in}

\title{\textbf{Experiment 5: Op-Amp Applications}}
\author{Abhimanyu Koushik \brak{\text{EE24BTECH11024}},\\Agamjot Singh \brak{\text{EE24BTECH11002}}\\IIT Hyderabad}
\date{March 3, 2025}

\begin{document}

\maketitle

\begin{abstract}
This experiment explores three key applications of operational amplifiers (op-amps): custom weighted summing and difference amplifiers, op-amp integrators, and precision rectifiers. The mathematical principles, circuit designs, and working mechanisms are analyzed. Experimental results validate the theoretical predictions and demonstrate the versatility of op-amps in signal processing.
\end{abstract}

\section{  Introduction}
Operational amplifiers (op-amps) are versatile components widely used in analog signal processing. This experiment focuses on three applications:
1. Custom weighted summing and difference amplifier.
2. Op-amp integrator.
3. Precision rectifier (super diode).

Each application is implemented using appropriate circuit designs to perform mathematical operations or signal conditioning tasks.

\section{ Custom Weighted Summing \& Difference Amplifier}

\textbf{Objective}
To implement mathematical functions such as:
\[
V_{\text{out}} = 2V_1 + V_2 - V_3
\]
\[
V_{\text{out}} = 2V_1 - V_3
\]

 \textbf{Circuit Design}
The circuit uses an inverting summing amplifier with carefully chosen resistors to achieve desired weighting coefficients. If a non-inverting input is required, a combination of inverting and summing amplifiers can be used.

 \textbf{Components Required}
- Op-amp (e.g., LM741, TL081)
- Resistors (precisely selected for weighting)
- DC power supply
- Function generator (for input signals)
- Oscilloscope

 \textbf{Equation Derivation}
For an inverting summing amplifier:
\[
V_{\text{out}} = -\left(\frac{R_f}{R_1} V_1 + \frac{R_f}{R_2} V_2 + \frac{R_f}{R_3} V_3 \right)
\]
By selecting appropriate resistor values (\(R_f, R_1, R_2, R_3\)), the desired coefficients for \(V_1, V_2, V_3\) can be achieved.

---

\section{  Op-Amp Integrator}

 \textbf{Objective}
To design a circuit that performs mathematical integration:
\[
V_{\text{out}} = -\frac{1}{RC} \int V_{\text{in}} \, dt
\]

 \textbf{Circuit Design}
The circuit uses an operational amplifier with a capacitor in the feedback path instead of a resistor. This configuration enables the op-amp to act as a continuous-time integrator.

 \textbf{Components Required}
- Op-amp (e.g., LM741, TL081)
- Resistor (\(R\))
- Capacitor (\(C\))
- DC power supply
- Function generator
- Oscilloscope

 \textbf{Working Principle}
The op-amp integrator converts a square wave input into a triangular wave output. It acts as a continuous-time integrator in signal processing applications.

---

\section{  Precision Rectifier (Super Diode)}

 \textbf{Objective}
To design a precision rectifier capable of rectifying small AC signals without the voltage drop issue of standard diodes.

 \textbf{Circuit Design}
The circuit uses an op-amp to control a diode for full-wave or half-wave rectification. The op-amp eliminates the \(0.7\,V\) threshold voltage of conventional diodes.

 \textbf{Components Required}
- Op-amp (e.g., LM358, TL081)
- Diode (e.g., 1N4148)
- Resistors
- AC signal generator
- Oscilloscope

 \textbf{Equation Derivation}
For a half-wave rectifier:
\[
V_{\text{out}} = 
    \begin{cases} 
      0 & V_{\text{in}} < 0 \\ 
      V_{\text{in}} & V_{\text{in}} > 0 
    \end{cases}
\]

For a full-wave rectifier, an additional summing stage is used to combine the positive portion with the inverted negative portion of the input signal.

---

\section{  Results and Observations}

 **Custom Weighted Summing Amplifier**
The experimental results confirmed that the circuit accurately implemented the mathematical functions:
1. \(V_{\text{out}} = 2V_1 + V_2 - V_3\)
2. \(V_{\text{out}} = 2V_1 - V_3\)

The output voltages measured on the oscilloscope matched theoretical predictions.

 **Op-Amp Integrator**
The integrator successfully converted square wave inputs into triangular wave outputs. The observed waveforms were consistent with theoretical expectations.

 **Precision Rectifier**
The precision rectifier effectively rectified small AC signals without any noticeable voltage drop. Both half-wave and full-wave configurations were tested and verified.

---

\section{  Conclusion}
This experiment demonstrated three key applications of operational amplifiers:
1. Weighted summing and difference amplifiers were implemented using resistor networks.
2. The op-amp integrator performed real-time integration of input signals.
3. The precision rectifier eliminated voltage drop issues associated with conventional diodes.

These applications highlight the versatility of op-amps in performing mathematical operations and signal conditioning tasks in electronic circuits.

---

\end{document}
